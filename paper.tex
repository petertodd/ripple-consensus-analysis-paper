\documentclass{article}
\usepackage{url}
\usepackage{mathtools}
\usepackage{tikz}
\usetikzlibrary{arrows,positioning}
\begin{document}
\title{Ripple Consensus Review}
\author{Peter Todd}
\date{FIXME}
\maketitle

\section{Activities performed}

The Ripple consensus review investigation had four major activities associated
with it:

\begin{enumerate}

    \item Reviewed whitepapers and development documentation at ripple.com

    \item Reviewed existing third-party critism.

    \item Setup and run a full node.

    \item Reviewed C++ implementation, specifically the 0.27.4 tag.\footnote{git commit 92812fe7239ffa3ba91649b2ece1e892b866ec2a}

\end{enumerate}


\section{Overall architecture}

Ripple has diverged significantly from the original concept of a decentralized
network representing money explicitly as debt relationship between
parties.\ref{btcmag-introducing-ripple} While the concept of a debt
relationship still exists in the form of trust lines between participants, the
bulk of the codebase and developer documentation now focuses\footnote{For
    instance the ``Tutorials'' section of the Ripple website only explains how
    to create transactions that modify the ledger; there is almost no information
available on how to actually use the trustlines feature.} on the use and
maintenance of a global ledger of transactions and account balances.
Additionally a native currency, XRP, has been added, which is used for
anti-spam transaction fees\footnote{A small amount of XRP is irrovocably
destroyed for every transaction.}, as well as to serve as a universal currency.


\subsection{Hashing and serialization}

Though various parts of the Ripple user API's and networking use industry
standard serialization formats like JSON and Google Protobuf, for
consensus-critical functionality Ripple uses a custom tag-length-value
serialization and hashing scheme. This is not unexpected as industry standard
serialization schemes rarely, if ever, account for the need to create hash
digests from represented objects.

\begin{equation}
    \textit{Serialize}(\text{obj}) = t_0 n_0 d_0 + \hdots + t_n n_n d_n
\end{equation}

Hashing of objects generally uses the following scheme, implemented by
STObject::getHash(), resulting in a $256\text{bit}$ digest:

\begin{equation}
    H(\text{obj}) = \text{SHA512}(p + \text{Serialize}(\text{obj}))[0:256\text{bits}]
\end{equation}

The prefix $p$ is per-object-type, guaranteeing that objects of different types
will always have a different hash.\footnote{This is commonly known as
\emph{tagged hashing} in the literature.} For objects containing signatures,
such as transactions, there is a similar but separate \emph{signing hash}
implemented by the function STObject::getSigningHash(). Unlike the standard
hash, the signing hash does not serialize signature fields.


\subsection{Transactions}

Unlike Bitcoin, transactions increment and decrement account balances; they do
not directly consume the output of other transactions. To prevent replay
attacks accounts and transaction have sequence numbers; a transaction is only
valid if the Sequence number is exactly one greater than the last-valided
transaction from the same account. Additionally an optional AccountTxnID field
is available for use within transactions; the transaction is only valid if the
sending account's previously-sent transaction matches the provided hash.

Authentication is performed with a simple signature scheme; a scripting system
is not available, nor is multisignature support. To implement the latter a
highly complex single-purpose scheme\ref{ripple-wiki-multisign} has been
proposed.


\subsection{Ledger}

Information on the exact structure of the Ripple ledger is somewhat spotty.


The Ripple consensus algorithm starts with a Unique Node List. 



\section{}


\bibliographystyle{plain}
\bibliography{paper}

\end{document}
